\documentclass{article}

\usepackage{geometry}
\usepackage{amsmath,fixltx2e,graphicx,float,ifpdf,hyperref}
\usepackage[USenglish]{isodate}
\geometry{letterpaper}

%% Create a command \e for scientific notation: MANTISSA\e{EXPONENT}
%% will display as MANTISSA x 10^{EXPONENT}
\providecommand{\e}[1]{\ensuremath{\times 10^{#1}}}

%% Create a command \HRule to create a 0.5mm wide horizontal rule across the
%% page
\newcommand{\HRule}{\rule{\linewidth}{0.5mm}}

%% Use the displaymath style in all math environments, including inline.
%% may make some lines taller but makes for more readable sums, integrations,
%% etc.
\everymath{\displaystyle}

\begin{document}

    \title{PHYS 331 Computational Project Proposal: Stellar Modeling: Polytropes}
        \author{Erin Conn \and Project Partner: Matthew Hurley}
        \date{}
    \maketitle

    \section{Background}

        Understanding stellar mechanics requires the use of mathematical models
        of the internal structure of stars. We understand stars to be nearly
        spherical collections of hot, compressed gas (fluid), so using what we
        know of stellar composition and Newtonian and fluid mechanics, we can
        create models for the pressure, density, temperature, and luminosity of
        stars. One of the simpler, but powerful, such models is the
        polytrope.\cite{hansen2004}

        In astrophysics, \textit{polytropes} are solutions to the Lane-Emden
        equation\cite{lane1870} for the gravitational potential of a spherically
        symmetric distribution of Newtonian, polytropic\footnote{obeys the
        relation \(PV^n=C\), where \(P=\text{pressure}\), \(V=\text{volume}\),
        \(n=\text{polytropic index}\), \(C=\text{constant}\)} fluid under its
        own gravitation\cite{leblanc2010}:

        \begin{equation}
            \label{eq:laneemden}
            \frac{1}{\xi^2}\frac{d}{d\xi}\left(\xi^2\frac{d\theta(\xi)}{d\xi}\right)=-\theta^n(\xi)
        \end{equation}

        The Lane-Emden equation is a dimensionless form of Poisson's equation
        for the divergence of a field:

        \begin{equation}
            \label{eq:genpoisson}
            \nabla^2\Phi = f
        \end{equation}

        where \(\Phi\) is a scalar potential function and \(f\) is a
        position-dependent density function. The Poisson equation is familiar in
        electrostatics\cite{griffiths2013} where it is commonly used to find the
        electric potential for a charge distribution.

        In the derivation in the next subsection, we get Poisson's equation in
        this form:

        \begin{equation}
            \label{eq:poisson}
            \frac{1}{r^2} \frac{d}{dr} \left( \frac{r^2}{\rho(r)}\frac{dP(r)}{dr} \right) = -4 \pi G\rho(r)
        \end{equation}

        where \(\rho\) is the density of a fluid, \(P\) is the pressure, \(G\)
        is Newton's gravitational constant, and \(r\) is the radial distance
        from the center of the distribution.

        This equation does not explicitly depend on the thermal structure and
        properties of the distribution (star), but the temperature dependence is
        embedded in the derivation of this equation - eq.~\ref{eq:hydroeq} below
        depends on temperature through an equation of state \(P(\rho,T,X_i)\),
        for example.

        Therefore in general Poisson's equation (eq.~\ref{eq:poisson}) cannot be
        solved independently of other energy-transport and conservation
        equations that are coupled through the thermal dependence. However, in a
        special case of the equation of state for \(P\) in which the pressure is
        independent of temperature, the mechanical and thermal structures of the
        star can be decoupled. This equation of state is usually
        written\cite[p.177]{leblanc2010}:

        \begin{equation}
            \label{eq:specstate}
            P(r) = K\rho_c^{\frac{n+1}{n}}\theta^{n+1}
        \end{equation}

        As we shall see in the following derivation, using this state equation
        and a little more mathematical manipulation yields the Lane-Emden
        equation(eq.~\ref{eq:laneemden}) above.

        \subsection{Derivation of the Lane-Emden
        Equation\cite[pp.176--179]{leblanc2010}}

            The Lane-Emden equation can be derived multiple ways; one way is from
            the equations for hydrostatic equilibrium and mass conservation:

            \begin{equation}
                \label{eq:hydroeq}
                \frac{dP(r)}{dr} = -\frac{\rho(r)GM(r)}{r^2}
                %\caption{Hydrostatic equilibrium of a spherical mass
                %distribution}
            \end{equation}

            \begin{equation}
                \label{eq:masscons}
                dM(r)=4\pi r^2\rho(r)dr \rightarrow \frac{dM(r)}{dr} = 4\pi r^2\rho(r)
                %\caption{Mass of a spherical shell of radius r and thickness dr
                %- Conservation of Mass}
            \end{equation}

            where \(P(r)\) is the gas pressure as a function of radial distance
            from the center of the distribution, \(\rho(r)\) is the gas density,
            \(G\) is Newton's gravitational constant, and \(M(r)\) is the mass
            enclosed within a sphere of radius \(r\).

            These equations can be related by multiplying eq.~\ref{eq:hydroeq}
            by \(r^2/\rho\):

            \[ \frac{r^2}{\rho(r)}\frac{dP(r)}{dr} = -\frac{r^2}{\rho(r)}
            \frac{\rho(r)GM(R)}{r^2} \]

            then differentiating with respect to \(r\):
            
            \[
            \frac{d}{dr}\left(\frac{r^2}{\rho(r)}\frac{dP(r)}{dr}\right)=-G\frac{dM(r)}{dr}
            \]

            Substituting in eq.~\ref{eq:masscons} we obtain Poisson's equation:

            \begin{equation}
                \frac{1}{r^2} \frac{d}{dr} \left( \frac{r^2}{\rho(r)}\frac{dP(r)}{dr} \right) = -4 \pi G\rho(r)
            \end{equation}

            Now, using the polytropic state equation:

            \begin{equation}
                \label{eq:polytropstate}
                    P=K\rho^{\frac{n+1}{n}}
            \end{equation}

            where \(n\) is called the \textit{polytropic index} and \(K\) is a
            constant, and defining a dimensionless function \(\theta(r)\):

            \begin{equation}
                \label{eq:thetar}
                \rho(r)=\rho_c\theta^n{r}
            \end{equation}

            where \(\rho_c\) is the central density of the star, we can rewrite
            the pressure as a function of \(\theta(r)\):

            \[
                P(r)=K\rho_c^{\frac{n+1}{n}}\theta^{n+1}(r)=P_c\theta^{n+1}(r)
            \]

            where \(P_c=K\rho_c^{\frac{n+1}{n}}\) is the central pressure of the
            star. Substituting this into eq.~\ref{eq:poisson}:

            \[
                K\rho_c^{\frac{n+1}{n}}\frac{1}{r^2}\frac{d}{dr}\left(\frac{r^2}{\rho_c\theta^n(r)}\frac{d\theta^{n+1}(r)}{dr}\right)=-4\pi
                G\rho_c\theta^n(r)
            \]

            This can be simplified a bit by realizing that
            \(\frac{d\theta^{n+1}(r)}{dr}=(n+1)\theta^n(r)\frac{d\theta(r)}{dr}\):

            \begin{equation}
                \label{eq:simplpois}
                \frac{(n+1)P_c}{4\pi
                G\rho_c^2}\frac{1}{r^2}\frac{d}{dr}\left(r^2\frac{d\theta(r)}{dr}\right)=-\theta^n(r)
            \end{equation}

            Since we defined \(\theta(r)\) as a dimensionless function, this
            equation requires that \(\frac{(n+1)P_c}{4\pi G\rho_c^2}\) has the
            dimension of length squared. For further simplification, we can
            define a new variable \(\alpha\) that depends on the polytropic
            index \(n\):

            \begin{equation}
                \label{eq:alpha}
                \alpha^2=\frac{(n+1)P_c}{4\pi G\rho_c^2}
            \end{equation}

            and a new dimensionless radius \(\xi\):

            \begin{equation}
                \label{eq:xi}
                \xi=\frac{r}{\alpha}
            \end{equation}

            Substituting this \(\xi\) into eq.~\ref{eq:simplpois} we finally
            obtain the Lane-Emden equation:

            \begin{equation}
                \frac{1}{\xi^2}\frac{d}{d\xi}\left(\xi^2\frac{d\theta(\xi)}{d\xi}\right)=-\theta^n(\xi)
            \end{equation}

            which can be rearranged into a \textit{\(\mathrm{2^{nd}}\) order homogeneous differential
            equation} for \(\theta(\xi)\):

            \begin{equation}
                \label{eq:lede}
                \frac{d^2\theta(\xi)}{d\xi^2}+\frac{2}{\xi}\frac{d\theta(\xi)}{d\xi}+\theta^n(\xi)=0
            \end{equation}

            This equation is only linear (and analytically solvable) for the
            case where \(n=0\) or \(n=1\), (an analytic solution exists for
            \(n=5\) as well, using some clever trickery), so we will be using
            numerical methods to find solutions to this equation.

    \section{Usefulness of the model}

        While it appears that the polytropic model sacrifices a great deal of
        information in the name of simplicity since no dependency on the star's
        thermal properties is assumed, these models have uses in interpreting
        stellar structures and are remarkably accurate for stars in hydrostatic
        equilibrium, such as main-sequence stars and white dwarfs, given their
        simplicity\cite[p.332]{hansen2004}.

        Carl Hansen \textit{et al} provide detailed examples of such
        applications in chapter 7 of \textit{Stellar Interiors}\cite{hansen2004}
        for the historically interesting Eddington standard model, and for
        approximations of white dwarfs, leading us to our proposal.

    \section{Proposal}

        We propose to use numerical methods covered in class to find solutions
        to the Lane-Emden equation\~ref{eq:laneemden} to create models of white
        dwarf stars and compare them with observed data from known dwarfs.

        White dwarfs are particularly good candidates for polytropic modeling
        because they are composed of degenerate electrons\footnote{all in the
        same energy state}. It is natural, and not wholly inaccurate, to assume
        they are composed of a completely degenerate\footnote{behaves as if the
        temperature is 0K} gas, which implies an equation of
        state\cite[pp.163-166]{hansen2004}:

        \begin{equation}
            \label{eq:whitedwarf}
            P=K\rho^{\gamma}
        \end{equation}

        where \(\gamma=5/3\) for a non-relativistic gas and \(\gamma=4/3\) for a
        completely relativistic gas.

        It is trivial to show that \(\gamma\) can be expressed in the form
        \(\frac{n+1}{n}\) and we obtain \(n=1.5\) and \(n=3\) polytropes for the
        non-relativistic and relativistic cases respectively.

        Hansen outlines several methods for solving the Lane-Emden equation
        numerically, two of which use methods we have covered in
        class\cite[pp.338-351]{hansen2004}: Fourth-order Runge-Kutta, which requires
        recasting the problem as a set of two first-order differential
        equations, or Newton-Raphson, which can be applied after linearizing the
        differential equation on a lattice. In fact, any of our techniques for
        solving systems of linear equations can be applied after linearizing the
        equations.


    \bibliographystyle{abbrv}
    \bibliography{poly}

\end{document}
