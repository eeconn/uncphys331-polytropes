\documentclass{article}

\usepackage{geometry}
\usepackage{amsmath,amsthm,fixltx2e,graphicx,float,ifpdf,hyperref}
\usepackage[USenglish]{isodate}
\geometry{letterpaper}

%% Create a command \e for scientific notation: MANTISSA\e{EXPONENT}
%% will display as MANTISSA x 10^{EXPONENT}
\providecommand{\e}[1]{\ensuremath{\times 10^{#1}}}

%% Create a command \HRule to create a 0.5mm wide horizontal rule across the
%% page
\newcommand{\HRule}{\rule{\linewidth}{0.5mm}}

%% Use the displaymath style in all math environments, including inline.
%% may make some lines taller but makes for more readable sums, integrations,
%% etc.
\everymath{\displaystyle}

%% Define theorem environment
\newtheorem{thm}{Theorem}

\begin{document}

    \section{Numerical Recipes:}

    \subsection{Standard BVP problem}

        Desire a solution to set of \(N\) coupled 1st order ODES satisfying
        \(n_1\) boundary conditions at starting point \(x_1\) and a remaining
        set of \(n_2=N-n_1\) conditions at final point \(x_2\). (Recall all DEs
        of order higher than 1 can be written as coupled sets of 1st order)

        Equations are:

        \begin{equation}
            \frac{dy_i(x)}{dx}=g_i(x,y_1,y_2,\dotsc ,y_n)\quad i=1,2,\dotsc ,N
        \end{equation}

        For a \textit{free boundary problem}:
        Only one boundary abcissa \(x_1\) is specified, while other boundary
        \(x_2\) is TBD so system has a solution satisfying total of \(N+1\)
        conditions. Add an extra constant dependent variable:

        \begin{equation}
            y_{N+1}\equiv x_2-x_1
        \end{equation}

        And another equation:

        \begin{equation}
            \frac{dy_{n+1}}{dx}=0
        \end{equation}

        And also define new independent variable \(t\):

        \begin{equation}
            t_{y_{N+1}}\equiv x-x_1,\quad 0\le t\le 1
        \end{equation}

        This is now a system of \(N+1\) differential equations for \(dy_i/dt\)
        in standard form with \(t\) varying between known limits 0 and 1.

        \begin{equation}
            \frac{dy_i}{dt} = g_i(t,x,y_1,y_2,\dotsc ,y_n),\quad i=1,2,\dotsc ,N+1
        \end{equation}

    \subsection{How does this apply to our problem?}

        We do not know what the radius \(\xi_2\) of the star is, only that
        \(\theta\) must go to 0 at some point. We therefore have a free-boundary
        problem with an unknown \(\xi_2\) and a known condition at that point.

        Our original differential equation, rearranged into homogenous form:

        \begin{equation}
            \frac{d^2\theta(\xi)}{d\xi^2} +
            \frac{2}{\xi}\frac{d\theta(\xi)}{d\xi} + \theta^n(\xi)=0
        \end{equation}

        Rearranged into a system of 1st-order equations:

        \begin{equation}
            \left\{
            \begin{array}{l}
                \frac{d\theta}{d\xi}=z \\
                \frac{dz}{d\xi}=-\frac{2}{\xi}z-\theta^n
            \end{array}\right.
        \end{equation}

        So I guess we add this new differential equation in?

        \[
            \left\{
            \begin{array}{l}
                x=\xi \\
                y_1=\theta \\
                y_2=z \\
                y_3=x_2-x_1 \\
                t=x-x_1
            \end{array}
            \right.
        \]

        \begin{equation}
            \left\{
            \begin{array}{l}
                \frac{dy_1}{dt}=y_2\frac{dx}{dt} \\
                \frac{dy_2}{dt}=-2x y_2\frac{dx}{dt}-? \\
                \frac{dy_3}{dt}=0
            \end{array}
            \right.
        \end{equation}

    \section{\href{http://www.vikdhillon.staff.shef.ac.uk/teaching/phy213/phy213_le.html}{Sheffield online lecture notes}}

    \subsection{\href{http://www.vikdhillon.staff.shef.ac.uk/teaching/phy213/phy213_polytropes.html}{Boundary values}}

    Lane-Emden in form:

    \begin{equation}
        \left(\frac{1}{\xi^2}\right)\frac{d}{d\xi}\left(\xi^2\frac{d\theta}{d\xi}\right)=-\theta^n
    \end{equation}

    To solve, need 2 boundary conditions. At center \(\left(\xi = 0 \right)\),
    \(\rho=\rho_c\) and hence \(\theta=1\). 2nd condition follows from equation
    of hydrostatic support in which \(\frac{M}{r^2}\rightarrow 0\) as
    \(r\rightarrow 0\). This means \(\frac{dP}{dr}=0\) at \(r=0\), and from the
    polytropic equation of state, \(\frac{d\theta}{d\xi}=0\) at \(\xi=0\).

    \subsection{How to solve with these conditions?}

    \emph{This looks like Euler's method, not the most accurate, but we can
    probably easily extend this to RK4.}

    Express Lane-Emden in form

    \begin{equation}
        \frac{d^2\theta}{d\xi^2} =
        -\left(\frac{2}{\xi}\frac{d\theta}{d\xi}\right)-\theta^n
        \label{eq:leshef}
    \end{equation}

    Step outward from radius from the center and evaluate density at each
    radius. Value of the density \(\theta_{i+1}\) given by value of density at
    the previous radius \(\theta_i\) plus the change in density at each step.

    \begin{equation}
        \theta_{i+1}=\theta_i+\Delta\xi \left(\frac{d\theta}{d\xi}\right)_{i+1}
    \end{equation}

    The rate of change of density with radius \(\frac{d\theta}{d\xi}\) is an
    unknown in the above equation. To determine \emph{its} value, use the same
    technique:

    \begin{equation}
        \left(\frac{d\theta}{d\xi}\right)_{i+1}=\left(\frac{d\theta}{d\xi}\right)_{i} +
        \Delta\xi\frac{d^2\theta}{d\xi^2}
    \end{equation}
    
    The \(\frac{d^2\theta}{d\xi^2}\) term is given by eq.~\ref{eq:leshef}:

    \begin{equation}
        \left(\frac{d\theta}{d\xi}\right)_{i+1} =
        \left(\frac{d\theta}{d\xi}\right)_i-\left(\frac{2}{\xi_i}\left(\frac{d\theta}{d\xi}\right)_i+\theta_i^n\right)\Delta\xi
    \end{equation}

    Numerical integration for a particular polytropic index \(n\) can proceed as
    follows:

    Starting at the center, at which values of \(\xi,\frac{d\theta}{d\xi},\text{
    and}\theta\) are known because of the boundary conditions given above,
    determine \(\left(\frac{d\theta}{d\xi}\right)_{i+1}\). This value is used to determine
    \(\theta_{i+1}\). The radius is incremented by adding \(\Delta\xi\text{ to}
    \xi\) and the process repeated until the surface is reached (when \(\theta\)
    becomes negative).

    In the online notes
    \(\Delta\xi=0.001\) and \(\xi_0=10\e{-5}\) to avoid singularity
    at origin.

    \section{Mass-Radius Relation}

        \begin{align*}
            M_*&=\int_0^{R_*}{4\pi r^2 \rho(r)}dr \\
               &=4\pi\alpha^3\rho_c\int_0^{\xi_0}{\xi^2\theta^n(\xi)}d\xi
        \end{align*}

        We'll need to  know \(\alpha\) and \(\rho_c\). We'll also need to use a
        numerical integrator for this part as well, since we only have a
        numerical solution.

        \begin{thm}
            Simpson's rule:

            Uses quadratic interpolation to numerically integrate over a region

            \[\int_a^b{f(x)}dx\approx\frac{b-a}{6}\left[f(a)+4f\left(\frac{a+b}{2}\right)+f(b)\right]\]
        \end{thm}

        \begin{align*}
            \alpha^2&=\frac{(n+1)P_c}{4\pi G\rho_c^2} \\
            P_c&=K\rho_c^{\frac{n+1}{n}}
        \end{align*}

        Does this mean we need to know K as well?


\end{document}
