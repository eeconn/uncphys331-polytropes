%%%%%%%%%%%%%%%%%%%%%%%%%%%%%%%%%%%%%%%%%
% Journal Article
% LaTeX Template
% Version 1.3 (9/9/13)
%
% This template has been downloaded from:
% http://www.LaTeXTemplates.com
%
% Original author:
% Frits Wenneker (http://www.howtotex.com)
%
% License:
% CC BY-NC-SA 3.0 (http://creativecommons.org/licenses/by-nc-sa/3.0/)
%
%%%%%%%%%%%%%%%%%%%%%%%%%%%%%%%%%%%%%%%%%

%----------------------------------------------------------------------------------------
%	PACKAGES AND OTHER DOCUMENT CONFIGURATIONS
%----------------------------------------------------------------------------------------

\documentclass[twoside]{article}

%\usepackage[sc]{mathpazo} % Use the Palatino font
%\usepackage[T1]{fontenc} % Use 8-bit encoding that has 256 glyphs
%\linespread{1.05} % Line spacing - Palatino needs more space between lines
%\usepackage{microtype} % Slightly tweak font spacing for aesthetics

\usepackage[hmarginratio=1:1,top=32mm,columnsep=20pt]{geometry} % Document margins
\usepackage{multicol} % Used for the two-column layout of the document
\usepackage{appendix} % Use to add an appendix page to the document
\usepackage[hang, small,labelfont=bf,up,textfont=it,up]{caption} % Custom captions under/above floats in tables or figures
\usepackage{booktabs} % Horizontal rules in tables
\usepackage{float} % Required for tables and figures in the multi-column environment - they need to be placed in specific locations with the [H] (e.g. \begin{table}[H])
\usepackage{hyperref} % For hyperlinks in the PDF
\geometry{letterpaper} % Scale for printing on 8.5"x11" paper

\usepackage{lettrine} % The lettrine is the first enlarged letter at the beginning of the text
\usepackage{paralist} % Used for the compactitem environment which makes bullet points with less space between them

\usepackage{abstract} % Allows abstract customization
\renewcommand{\abstractnamefont}{\normalfont\bfseries} % Set the "Abstract" text to bold
\renewcommand{\abstracttextfont}{\normalfont\small\itshape} % Set the abstract itself to small italic text

\usepackage{titlesec} % Allows customization of titles
\renewcommand\thesection{\Roman{section}} % Roman numerals for the sections
\renewcommand\thesubsection{\Roman{subsection}} % Roman numerals for subsections
\titleformat{\section}[block]{\large\scshape\centering}{\thesection.}{1em}{} % Change the look of the section titles
\titleformat{\subsection}[block]{\large}{\thesubsection.}{1em}{} % Change the look of the section titles

\usepackage{fancyhdr} % Headers and footers
\pagestyle{fancy} % All pages have headers and footers
\fancyhead{} % Blank out the default header
\fancyfoot{} % Blank out the default footer
\fancyhead[C]{Polytropes $\bullet$ April 21, 2014 } % Custom header text
\fancyfoot[RO,LE]{\thepage} % Custom footer text

%----------------------------------------------------------------------------------------
%	TITLE SECTION
%----------------------------------------------------------------------------------------

\title{\vspace{-15mm}\fontsize{24pt}{10pt}\selectfont\textbf{Polytropes and
Models of White Dwarf Stars}} % Article title

\author{
\large
\textsc{Erin Conn, Matthew Hurley}\\[2mm] % Your name
\normalsize University of North Carolina at Chapel Hill \\ % Your institution
\vspace{-5mm}
}
\date{}

%----------------------------------------------------------------------------------------

\begin{document}

\maketitle % Insert title

\thispagestyle{fancy} % All pages have headers and footers

%----------------------------------------------------------------------------------------
%	ABSTRACT
%----------------------------------------------------------------------------------------

\begin{abstract}

\noindent 

\end{abstract}

%----------------------------------------------------------------------------------------
%	ARTICLE CONTENTS
%----------------------------------------------------------------------------------------

\begin{multicols}{2} % Two-column layout throughout the main article text

\section{Introduction}

\lettrine[nindent=0em,lines=2]{U}nderstanding stellar mechanics requires the use
of mathematical models of the internal structure of stars. We understand stars
to be nearly spherical collections of hot gas held together by self-gravitation
and holding themselves up against gravitational collapse by fluid and radiation
pressure. Equations from Newtonian gravitational and fluid mechanics allow us to
create models for the how the pressure, density, and temperature of stars varies
with their mass and size, but they are complicated and highly
coupled.\cite{hansen2004} Additionally, as mass and density increase, quantum
and relativistic effects become important.

Prior to the development of advanced electronic computers, solutions to these
equations were difficult to impossible to compute. In 1870, American physicist
Jonathan Homer Lane proposed a simplified model\cite{lane1870} by assuming the
gas pressure depends only on the density of the gas (a \emph{polytropic fluid}),
eliminating any explicit dependence on temperature and decoupling the equations
for pressure, temperature, and density.

    \begin{equation}
        \label{eq:polystate}
        P(r) = K\rho^{1+1/n}
    \end{equation}
    
Later Swiss physicist Robert Emden formalized the model in the dimensionless
differential equation that bears their names, the Lane-Emden equation, whose
derivation can be found in the appendix.

    \begin{equation}
        \label{eq:le}
        \frac{1}{\xi^2}\frac{d}{d\xi}\left(\xi^2\frac{d\theta}{d\xi}\right)+\theta^n=0
    \end{equation}

where \(\xi\) is a dimensionless function of the radius, \(\theta\) is a
dimensionless function relating density and pressure, and n is the
\emph{polytropic index} of the fluid.  Solutions to this equation are called
\emph{polytropes}.

While Lane's polytropic simplification seemed unrealistic for most situations
conceivable at the time, the vast simplification of its solutions over more
realistic models proved a generous return on the trade with results that were
still close enough to observed data to be very useful. Closed-form
solutions\cite{leblanc2010} can
be found readily found for polytropic indices \(n=0\) and \(n=1\), and with some
algebraic substitution another one can be found for \(n=5\) (which has an
infinite radius and will not be discussed further), and numerical solutions can
be found using techniques known at the time of Lane's original publication.

In the 20th century new utility was found for this model with the development of
quantum mechanics and the discovery of \emph{white dwarfs}, stellar remnants
composed of extremely dense gas. The electron density in white dwarfs approaches
the density of available quantum energy states; and since the Pauli exclusion
principle disallows multiple electrons from occupying the same quantum
state\cite[p.216]{griffithsqm}, this results in what is called \emph{electron
degeneracy pressure} which is the main force opposing the dwarf's own gravity
since white dwarfs are no longer actively undergoing
fusion.\cite[pp.163--166]{hansen2004} White dwarfs are therefore composed of
nearly fully \emph{degenerate matter}. Serendipitously, fully degenerate matter
has the following equation of state:

    \begin{equation}
        \label{eq:degenstate}
        P(r)=K\rho(r)^{\gamma}
    \end{equation}

identical in form to eq.~\ref{eq:polystate}, \emph{i.e.}, it is a polytropic gas
with \(\gamma\equiv 1+\frac{1}{n}\)\footnote{not to be confused with the
relativistic \(\gamma\)--factor}!

White dwarfs are therefore ideally suited to modeling with polytropes. The
value of the polytropic index is found from fluid and quantum mechanical
relations. In fact, there are two polytropic indices that apply for degenerate
gases; in lower energy states the electrons have non-relativistic momenta and
have a \(\gamma\) index of \(5/3\), which corresponds to a polytropic index of
\(n=1.5\). As density increases, more of the electrons occupy higher energy
states with higher momenta, and relativistic effects prevail, resulting in a
\(\gamma\) index of \(4/3\), corresponding to a polytropic index of \(n=3\). 


We used numerical integration methods to solve the Lane-Emden equation for these
polytropic indices, found the relationship between the mass and radius of the
resulting polytropes, and compared results with observed data for several known
white dwarfs.
%------------------------------------------------

\section{Methods}

The Lane-Emden equation is a 2nd-order nonlinear (for values of \(n\) other than
0 or 1) differential equation in one variable (\(\xi\)). It is not analytically solvable in most cases,
but solutions to initial and boundary value problems for this equation can be
found using numerical integration.

The major challenge in solving the equation lay in identifying the boundary
conditions. 
%------------------------------------------------

\section{Results}

We obtained the following parameters for nonrelativistic (\(n=1.5\)) and
relativistic (\(n=3\)) polytropes using a Runge-Kutta scheme in Matlab:

\begin{table}[H]
\caption{Solutions obtained with Runge-Kutta}
\centering
\begin{tabular}{c | c c c}
\toprule
%\multicolumn{2}{c}{Name} \\
%\cmidrule(r){1-2}
\(n\) & \(\xi_f\) & \(\theta'(\xi_f)\) & \(\rho_c\) \\
\midrule
1.5 & 3.6838 & -0.2033 & 5.9907 \\
3 & 6.8968 & -0.0424 & 54.1825 \\
\bottomrule
\end{tabular}
\end{table}


Compared with the parameters outlined in \textit{Stellar
Interiors}\cite{hansen2004} any discrepancies can be attributed to rounding
differences.

\begin{table}[H]
\caption{Parameters for \(n=1.5\) and \(n=3\) polytropes\cite{hansen2004}}
\centering
\begin{tabular}{c | c c c}
\toprule
%\multicolumn{2}{c}{Name} \\
%\cmidrule(r){1-2}
\(n\) & \(\xi_f\) & \(\theta'(\xi_f)\) & \(\rho_c\) \\
\midrule
1.5 & 3.6538 & -0.20330 & 5.991 \\
3 & 6.8969 & -0.04243 & 54.1825 \\
\bottomrule
\end{tabular}
\end{table}

%------------------------------------------------

\section{Discussion}

\end{multicols}
%----------------------------------------------------------------------------------------
%	REFERENCE LIST
%----------------------------------------------------------------------------------------

\bibliographystyle{abbrv}
\bibliography{poly}
%----------------------------------------------------------------------------------------
%   APPENDICES
%----------------------------------------------------------------------------------------
\appendix
\appendixpage
\section{Derivation of the Lane-Emden Equation\cite[pp.176--179]{leblanc2010}} 

The Lane-Emden equation can be derived multiple ways; one way is from the
equations for hydrostatic equilibrium and mass conservation:

\begin{equation}
    \label{eq:hydroeq}
    \frac{dP(r)}{dr} = -\frac{\rho(r)GM(r)}{r^2}
    \end{equation}

    \begin{equation}
    \label{eq:masscons}
    dM(r)=4\pi r^2\rho(r)dr \rightarrow \frac{dM(r)}{dr} = 4\pi r^2\rho(r)
    \end{equation}

where \(P(r)\) is the gas pressure as a function of radial distance from the
center of the distribution, \(\rho(r)\) is the gas density, \(G\) is Newton's
gravitational constant, and \(M(r)\) is the mass enclosed within a sphere of
radius \(r\).

These equations can be related by multiplying eq.~\ref{eq:hydroeq} by
\(r^2/\rho\):

\[ \frac{r^2}{\rho(r)}\frac{dP(r)}{dr} = -\frac{r^2}{\rho(r)}
\frac{\rho(r)GM(R)}{r^2} \]

then differentiating with respect to \(r\):
            
\[
\frac{d}{dr}\left(\frac{r^2}{\rho(r)}\frac{dP(r)}{dr}\right)=-G\frac{dM(r)}{dr}
\]

Substituting in eq.~\ref{eq:masscons} we obtain Poisson's equation for gravitational potential:

\begin{equation}
    \label{eq:poisson}
    \frac{1}{r^2} \frac{d}{dr} \left( \frac{r^2}{\rho(r)}\frac{dP(r)}{dr} \right) = -4 \pi G\rho(r)
\end{equation}

Now, using the polytropic state equation:

\begin{equation}
    \label{eq:polytropstate}
    P=K\rho^{\frac{n+1}{n}}
\end{equation}

where \(n\) is called the \textit{polytropic index} and \(K\) is a constant, and
defining a dimensionless function \(\theta(r)\):

\begin{equation}
    \label{eq:thetar}
    \rho(r)=\rho_c\theta^n{r}
\end{equation}

where \(\rho_c\) is the central density of the star, we can rewrite the pressure
as a function of \(\theta(r)\):

            \[
                P(r)=K\rho_c^{\frac{n+1}{n}}\theta^{n+1}(r)=P_c\theta^{n+1}(r)
            \]

            where \(P_c=K\rho_c^{\frac{n+1}{n}}\) is the central pressure of the
            star. Substituting this into eq.~\ref{eq:poisson}:

            \[
                K\rho_c^{\frac{n+1}{n}}\frac{1}{r^2}\frac{d}{dr}\left(\frac{r^2}{\rho_c\theta^n(r)}\frac{d\theta^{n+1}(r)}{dr}\right)=-4\pi
                G\rho_c\theta^n(r)
            \]

            This can be simplified a bit by realizing that
            \(\frac{d\theta^{n+1}(r)}{dr}=(n+1)\theta^n(r)\frac{d\theta(r)}{dr}\):

            \begin{equation}
                \label{eq:simplpois}
                \frac{(n+1)P_c}{4\pi
                G\rho_c^2}\frac{1}{r^2}\frac{d}{dr}\left(r^2\frac{d\theta(r)}{dr}\right)=-\theta^n(r)
            \end{equation}

            Since we defined \(\theta(r)\) as a dimensionless function, this
            equation requires that \(\frac{(n+1)P_c}{4\pi G\rho_c^2}\) has the
            dimension of length squared. For further simplification, we can
            define a new variable \(\alpha\) that depends on the polytropic
            index \(n\):

            \begin{equation}
                \label{eq:alpha}
                \alpha^2=\frac{(n+1)P_c}{4\pi G\rho_c^2}
            \end{equation}

            and a new dimensionless radius \(\xi\):

            \begin{equation}
                \label{eq:xi}
                \xi=\frac{r}{\alpha}
            \end{equation}

            Substituting this \(\xi\) into eq.~\ref{eq:simplpois} we finally
            obtain the Lane-Emden equation:

            \begin{equation}
                \frac{1}{\xi^2}\frac{d}{d\xi}\left(\xi^2\frac{d\theta(\xi)}{d\xi}\right)=-\theta^n(\xi)
            \end{equation}


\end{document}
