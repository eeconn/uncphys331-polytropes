\documentclass{beamer}

\usepackage{amsmath}

\usetheme{Dresden}
\usecolortheme{default}

\title[Polytropes] % OPTIONAL: only for long titles
{Polytropic Models of White Dwarfs}
\subtitle{Examples for use of the beamer package}
\author[Conn, Hurley] % OPTIONAL: for multiple authors
{Erin Conn \and Matthew Hurley}
\subject{Polytropes}

\AtBeginSection[]
{
        \begin{frame}<beamer>
            \frametitle{Table of Contents}
            \tableofcontents[currentsection]
        \end{frame}
}


\AtBeginSubsection[]
{
    \begin{frame}<beamer>
        \frametitle{Table of Contents}
        \tableofcontents[currentsection,currentsubsection]
    \end{frame}
}

\begin{document}

    \frame{\titlepage}

    \section{Theory}

        \subsection{Polytropes}

        \begin{frame}
            \frametitle{What are polytropes?}

            Solutions to...

            \textbf{The Lane-Emden Equation}

            \begin{equation}
                \frac{1}{\xi^2}\frac{d}{d\xi}\left(\xi^2\frac{d\theta}{d\xi}\right)=-\theta^n(\xi)
            \end{equation} 

            A dimensionless, 2nd order nonlinear differential equation relating the
            pressure of a spherically-symmetric gas distribution to the radius.

        \end{frame}

        \begin{frame}
            \frametitle{Definitions}

            \begin{definition}
                \alert{Polytropic process} - Thermodynamic process that obeys the relation
                
                \[Pv^n=C\]
            \end{definition} 

            \begin{definition}
                \alert{Polytropic index} - Constant that relates pressure of a polytropic fluid to its volume (density). It may be any real number.
            \end{definition}

        \end{frame}

%        \begin{frame}
%        %commenting this frame out for now; it's making pdflatex choke for some reason
%            \frametitle{Derivation - 1}
%
%            Can be derived multiple ways. From laws of mass conservation and
%            hydrostatic equilibrium:
%
%            \begin{align*}
%                dM(r) &= 4\pi r^2\rho(r)dr \rightarrow \frac{dM(r){dr}=4\pi r^2\rho(r) \\
%                \frac{dP(r)}{dr} &= -\frac{\rho(r)GM(r)}{r^2}
%            \end{align*}
%
%            These equations are related by multiplying the hydrostatic equation by
%            \(r^2/\rho\) and differentiating:
%
%            \[\frac{d}{dr}\left(\frac{r^2}{\rho(r)}\frac{dP(r)}{dr}\right)=-G\frac{dM(r)}{dr}\]
%
%            Yielding Poisson's equation for gravity:
%
%            \[\frac{1}{r^2}\frac{d}{dr}\left(\frac{r^2}{\rho(r)}\frac{dP(r)}{dr}\right)=-4\piG\rho(r)\]
%
%        \end{frame}

        \subsection{White Dwarfs}

        \begin{frame}
            \frametitle{Placeholder}

        \end{frame}

    \section{Methods}

        \begin{frame}
            \frametitle{Alternate form of the Lane-Emden Equation}

            \[\frac{d^2\theta}{d\xi^2}+\frac{2}{\xi}\frac{d\theta}{d\xi}=-\theta^n(\xi)\]

        \end{frame}

        \begin{frame}
            \frametitle{Translating to a system of 1st order equations}

            \begin{align*}
                \phi &= \frac{d\theta}{d\xi} \\
                \frac{d\phi}{d\xi} &= -\frac{2}{\xi}\phi - \theta^n
            \end{align*}

        \end{frame}

    \section{Results}

        \begin{frame}
            \frametitle{Placeholder}

        \end{frame}

    \section{Discussion}

        \begin{frame}
            \frametitle{Questions?}

        \end{frame}

\end{document}
